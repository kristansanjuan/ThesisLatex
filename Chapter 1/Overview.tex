	\section{Overview}
	As the Philippines continues to modernize and urbanize, waste management has emerged as a critical environmental issue, affecting both rural and urban communities. According to Statista.com, the country produced over 59.24 thousand tons of waste per day in 2022. The combination daily waste production, rapid population growth, and lack of awareness regarding waste segregation has significantly contributes to environmental degradation. 
	
	This thesis proposes using technology to address waste management challenges, particularly waste segregation. By utilizing the widespread use of smartphones among Filipinos, this study aims to enhance public awareness and improve waste management practices. The proposed solution is the development of an app that enables users to accurately identify and categorize waste items, thereby facilitating proper waste segregation.
	
	The app employs advanced image recognition technology to scan waste items and determine their classification, such as biodegradable, non-biodegradable, recyclable, hazardous, or non-trash. It provides real-time feedback on the appropriate waste bin for each item, simplifying the segregation process and educating users on proper waste management.
	
	This thesis will explore the development, deployment, and potential impact of the waste management app, focusing on its technological framework, user interface design, machine learning models for image recognition, and overall effectiveness in promoting sustainable waste disposal habits.