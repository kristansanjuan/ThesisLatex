\section{Background of the Study}
Waste management has been acknowledged since ancient times, as demonstrated by early cities like Rome and Athens. Nevertheless, this mainly focus in only removing the waste rather than recycling or segregation. With time, especially during industrialization and city growth processes, there was an increase in quantities of wastes leading to more systematic waste management approaches. It was only the end of the 20th century that waste categorization emphasizing on biodegradable, non-biodegradable, and recyclable materials became importance hence transitioning towards a recycling and usability in modern waste management systems.

In today's world, waste management has become a critical global issue due to rapid increase in urbanization, industrialization, and population growth. Especially in the Philippines, where it has been reported to increase more than 59.24 thousand tons per day in 2022 \parencite{Statista2022}. Poor waste segregation, on one hand, contributes much to environmental degradation through soil and marine pollution. Various policies have been implemented to encourage proper waste disposal. However, public awareness and participation in proper segregation practices are still at a minimal rate.

Since mobile phone technology is increasingly becoming parts of people's live, it can provide a way to address such practical issues like waste management. This research proposes that a mobile application which will use an image recognition for sorting waste into biodegradable, non-biodegradable, and recyclable should be developed to help improve waste segregation practices. Through technology usage, this research aims to encourage people to properly segregate there waste and create an environmentally conscious behaviors.

Previous studies have explored various waste management approaches, such as manual classification, recycling programs, and automated sorting systems. However, research on mobile applications for real-time waste classification using image recognition is limited. For instance, a study by Malik et al. (2022) demonstrated the potential of using convolutional neural networks (CNNs) for classifying waste materials through images, showcasing effectiveness of deep learning models in waste sorting \parencite{Malik2022} Additionally, a study by Ahmad Faudzi et al. (2023) demonstrate that effective user experience and interaction design in mobile applications could enhance user engagement and positively influence behavior change \parencite{AhmadFaudzi2023}. Despite these advancements, a comprehensive waste classification system integrating real-time image recognition, particularly for mobile devices, is still lacking. This study aims to address this gap by developing an educational and practical mobile application for daily waste segregation purposes.