\section{Rationale}

	In today’s world, waste generation is gradually increasing and it has become a significant environmental problem globally. Improper waste management leads to many environmental and health issues including pollution, spread of diseases, depletion of natural resources, and habitat destruction. It also contributes to climate change by increasing greenhouse gas emissions from decomposing of waste in landfills. Despite the implementation of numerous plans and technologies to address this issue, there remains a considerable lack of public awareness regarding proper waste segregation. Many individuals lack knowledge or tools to correctly correctly identifies whether an item is recyclable, non-biodegradable, or biodegradable, leading to improper disposal practices. Our research revealed that existing apps often provide general information or require manual input, which can be time consuming and error-prone.

	To address this issue, we propose an innovative solution to enhance public awareness and engagement in proper waste segregation by simply taking a simple photo, this solution will empower individuals to make the decision about waste disposal. Thus leading to improved cycling rates, reduced environmental and health impact and a more sustainable future.
	
	The rationale behind this study is to create an accessible, user-friendly tool that closes the lack of knowledge in waste management, enabling people to easily dispose their waste properly.