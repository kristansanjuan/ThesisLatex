\section{Significance}
	The programmed system holds significant potential as it seeks to address real-world environmental challenges through technological innovation. The development of a system capable to categorizing waste as biodegradable, non-biodegradable and recyclable will be instrumental in enhancing of sorting practices, thereby contributing to effective waste segregation. The following outlines the benefits of the system to various groups.
	
\begin{itemize}
	\item \textbf{To students}: The system will enrich students' learning experiences in waste management, aiding their understanding of waste segregation and fostering greater environmental awareness.
	
	\item \textbf{To educators}: This tool will serve as a valuable resource for teachers, particularly those instructing in environmental science or related subjects. It can be integrated into teaching strategies to enhance engagement and effectively promote proper waste segregation practices.
	
	\item \textbf{To individuals}: The system will increase individual awareness regarding waste segregation, encouraging healthy environmental practices beginning at home, which can have a positive ripple effect on communities.
	
	\item \textbf{To environmentalists}: This system will support environmental advocates in promoting initiatives related to proper waste management, providing a practical tool for awareness campaigns.
	
	\item \textbf{To waste management professionals}: The system will aid in the sorting and collection of waste, improving recycling processes in facilities, and ensuring the proper handling of waste disposal.
	
	\item \textbf{To government authorities}: The system will be valuable for government efforts to promote waste sorting, enhancing waste management and recycling programs within communities.
	
	\item \textbf{To future researchers}: This system may inspire future researchers to build upon this work, developing new technological solutions that contribute to environmental sustainability.
\end{itemize}
	