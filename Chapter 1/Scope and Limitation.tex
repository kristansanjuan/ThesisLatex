\section{Scope and Limitation}

	\subsection{Scope}
	\begin{itemize}
		\item \textbf{Research Focus}: The study aims to develop and evaluate a mobile application designed to assist users in waste segregation by recognizing and categorizing waste as biodegradable, non-biodegradable, or recyclable. The app will utilize image recognition technology in alongside with a comprehensive database of waste items to ensure accurate classification. Additionally, the effectiveness of the user interface in guiding users and enhancing environmental awareness through educational content will be assessed.
		
		\item \textbf{Geographical Scope and Time Frame}: The study will be conducted in the Philippines, focusing on waste items commonly encountered in daily life across different regions. Over a six-month period, the app will be tested with real-world waste data collected from various locations within the country, ensuring that the model is trained with waste types that users in the Philippines typically interact with. Upon completion, the app will be made accessible to a broader audience through Google Play store, extending its practical implication beyond the study;s local scope.
		
		\item \textbf{Population and Sample}: The app will be initially distributed online, allowing a broad range of users across the Philippines to interact with it. By gathering feedback from users through online reviews, insights into app's functionality and usability will be gained. This feedbacks will be crucial in refining the app before its eventual release.
		
		\item \textbf{Variables and Concepts}: The key variables include the accuracy of the app's image recognition model in correctly classifying waste into their categories as well as its usability and impact on users'  behavior regarding waste segregation. The apps educational features will be evaluated for their influence on increasing environmental awareness. The concept of environmental awareness will be explored to evaluate how the app's education content influences users' understanding and concern about waste management. Technological reliability will be measured by the app's performance under various real-world scene particularly such as lighting and background noise.
		
		\item \textbf{Methodology}: This study will develop and evaluate a mobile application for waste segregation using image recognition technology. The application will be designed with a user-friendly interface in Android studio and will integrate a machine learning model trained with a vast amount of waste images collected from daily user interactions in the Philippines. The model is trained in Google's AI platform, to ensure that it will correctly classify waste into biodegradable, non-biodegradable, and recyclable. Evaluation of the app's effectiveness will involve usability testing and assessments under various real-world conditions such as lightning and background noise. User feedback will collected through online surveys to refine the app based on practical usage. Additionally, the study will measure changes in waste segregation behavior and environmental awareness to assess the app's impact on user and its contribution to improved waste management practices.
	\end{itemize}
	
	\subsection{Limitation}
	\begin{itemize}
		\item \textbf{Methodological Limitations}: The image recognition technology may face challenges in accurately classifying waste items that are damaged, dirty, or unrecognizable. This could result in incorrect classifications and recommendations, potentially affecting the overall effectiveness of the app.
		
		\item \textbf{Sample Size and Selection}: Since the app will be released broadly and not tested on a controlled sample. This may limit the ability to generalize findings based on the feedback received from a diverse user base.
		
		\item \textbf{Data Availability}: The app’s database of waste items may initially be incomplete, potentially leading to misclassification of new or uncommon items. Additionally, accurate and comprehensive information from local recycling facilities is necessary to ensure that the app’s recommendations align with local recycling practices and policies
		
		\item \textbf{External Factors}: Changes in local or national waste management policies during the study could affect the app’s recommendations, leading to inconsistencies. Moreover, factors such as users' access to waste segregation resources, like appropriate bins and facilities, may influence how effectively users can implement.
		
		\item \textbf{Time Constraints}: The one-year timeframe for this study may limit the depth of testing and refinement. This constraint might impact the app’s ability to fully develop its database and enhance image recognition accuracy, potentially affecting the final version’s performance and completeness.
	\end{itemize}