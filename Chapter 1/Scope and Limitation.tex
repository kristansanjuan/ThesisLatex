\section{Scope and Limitation}

	\subsection{Scope}
	\begin{itemize}
		\item \textbf{Research Focus}: The study aims to develop and evaluate a mobile application designed to assist users in waste segregation by recognizing and categorizing waste as biodegradable, non-biodegradable, or recyclable. The app will utilize image recognition technology in alongside with a comprehensive database of waste items to ensure accurate classification. Additionally, the effectiveness of the user interface in guiding users and enhancing environmental awareness through educational content will be assessed.
		
		\item \textbf{Geographical Scope and Time Frame}: The research will be conducted in the Philippines, specifically targeting residents of Malolos, Bulacan. The study will span six months, covering both semesters of the academic year. This includes testing the app across different regions within the Philippines to ensure its functionality and adaptability.
		
		\item \textbf{Population and Sample}: The target population comprises residents of Malolos, Bulacan. A sample size of 50 individuals will be selected to provide insights into the app’s usability and effectiveness across diverse communities.
		
		\item \textbf{Variables and Concepts}: Key variables include the app’s accuracy in waste classification and the impact of its educational features on users' waste disposal behaviors. Concepts under examination include waste segregation, environmental awareness, and the role of mobile technology in promoting effective waste management.
		
		\item \textbf{Methodology}: The study involves developing a mobile application using Google’s AI platform for image recognition. The model will be trained with images of waste to ensure accurate classification. Android Studio will be used to create a user-friendly interface. The app’s performance will be tested under real-world conditions such as varying lighting and backgrounds. Usability testing and feedback will evaluate the app’s impact on users' waste habits and environmental awareness. The study aims to provide a foundational framework for future waste management technologies.
	\end{itemize}
	
	\subsection{Limitation}
	\begin{itemize}
		\item \textbf{Methodological Limitations}: The image recognition technology may struggle with accurately classifying waste items that are damaged, dirty, or otherwise unrecognizable. This could result in incorrect classifications and recommendations.
		
		\item \textbf{Sample Size and Selection}: With a sample size of only 50 participants, the study’s findings may not fully represent the diverse population of the Philippines. Focusing on Malolos, Bulacan, may also limit insights into waste management practices in more remote areas.
		
		\item \textbf{Data Availability}: The app’s database of waste items may initially be incomplete, potentially leading to misclassification of new or uncommon items. Inadequate documentation of local recycling facilities or waste management systems could also limit the app’s effectiveness in certain regions.
		
		\item \textbf{External Factors}: Changes in local or national waste management policies during the study could affect the app’s recommendations, leading to inconsistencies. Additionally, variations in users' access to waste segregation resources could impact the app's effectiveness.
		
		\item \textbf{Time Constraints}: The six-month timeframe may limit the depth of the app’s testing and refinement. This could affect the expansion of the database and the accuracy of the image recognition model, potentially resulting in a less comprehensive final product.
	\end{itemize}